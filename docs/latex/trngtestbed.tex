\subsection{TRNG Testbed}

At a fundamental level, every entropy source needs to be verifiable as a standalone implementation as well as be 
integrable into a given platform. What the TRNG Testbed IP seeks to do is:

\begin{enumerate}
    \item Provide standard interfaces to all entropy sources, so that testing and integration is seamless.
    \item Extend the interfaces to allow an instantiation of several entropy sources of various types
        for evaluation of several implementations, either of like entropy sources or of different entropy sources
        on the same platform.
\end{enumerate} 

Two implementations are provided with the open-source catalog; the TRNG Testbed as a platform, and as an IP core. The 
platform variant targeting the Digilent Genesys 2 is designed to provide 8 different instantiations of entropy sources
that are sampled at implementation-defined clock frequencies. Once connected to a host computer and programmed, the 
onboard processor listens over UART for the desired test to be run, either the million-consecutive-samples test or the 
startup test. The platform then takes the generated entropy samples and pipes them through a sample counter to the 
onboard DDR3 RAM. The NIST SP 800-90B minimum required number of consecutive samples is collected, and then streamed 
back to the host computer to be analyzed.

The IP core variant is designed for integration into a processor-based system on FPGA. The vast majority of TRNG 
applications are within processor-based designs, usually as a part of an algorithm such as key generation within 
cryptography like ML-KEM, simulations of complex and random natural phenomena, and more. As a result, the IP core 
variant of the TRNG Testbed is designed with an AXI4 Lite interface, maintaining compatibility with AMD Vivado block
design workflow. Additionally, because the TRNG Testbed is designed for entropy source instrumentation, an additional 
AXI interface will be implemented into the final, eventual implementation in order to provide a direct memory access
(DMA) feature, allowing the NIST SP 800-90B minimum required number of samples to be stored and read out by the 
processor from a designated memory region.

The TRNG Testbed IP core provides several configuration options at synthesis-time and at run-time. Firstly, a 
configurable number of different types of supported entropy sources can be instantiated through the top level generics.
By making the entropy source lineup configurable, users can instantiate several different types of entropy sources,
or several copies of the same entropy source, which allows for side-by-side comparison between two different entropy 
sources, or identifying the upper and lower bound of performance for the same entropy source. In addition, the IP core
allows selection of desired clock frequency. Several TRNGs have a range of clock frequencies the source authors have 
identified as optimal, thus the dynamic selection of clock frequency has the added benefit of providing local, per-
entropy-source optimization, as well as the ability to characterize min-entropy over a range of frequencies.

\begin{figure}
\centering
\includegraphics[width=0.45\textwidth]{figures/drawio/trngtestbed_toplevel.png} % Adjust width as needed
\caption{Architecture diagram of TRNG Testbed IP Core.}
\label{fig:trngtestbed_toplevel}
\end{figure}

The wide range of features provided by the TRNG Testbed are intended to support investigation, instrumentation,
and demonstration of a wide variety of entropy sources in a relevant application space. By standardizing the interfaces,
entropy sources become essentially plug-and-play within the TRNG Testbed or in other applications with a matching
interface. NIST SP 800-90B defines a set of standard entropy evaluation mechanisms, so in order to support min-entropy
estimation of a demonstration system, the TRNG Testbed provides the instrumentation necessary to perform this evaluation.
Finally, integration with a processor as a standard peripheral IP core defines a realistic and relevant demonstration
platform that will aid in furthering research and development of these entropy sources.