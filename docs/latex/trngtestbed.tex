\subsection{TRNG Testbed}

At a fundamental level, every entropy source needs to be verifiable as a standalone implementation as well as be 
integrable into a given platform. What the TRNG Testbed IP seeks to do is:

\begin{enumerate}
    \item Provide standard interfaces to all entropy sources, so that testing and integration is seamless.
    \item Extend the interfaces to allow an instantiation of several entropy sources of various types
        for evaluation of several implementations, either of like entropy sources or of different entropy sources
        on the same platform.
\end{enumerate}

Two implementations are provided with the open-source catalog; the TRNG Testbed as a platform, and as an IP core. The 
platform variant targeting the Digilent Genesys 2 is designed to provide 8 different instantiations of entropy sources
that are sampled at implementation-defined clock frequencies. Once connected to a host computer and programmed, the 
onboard processor listens over UART for the desired test to be run, either the million-consecutive-samples test or the 
startup test. The platform then takes the generated entropy samples and pipes them through a sample counter to the 
onboard DDR3 RAM. The NIST SP 800-90B minimum required number of consecutive samples is collected, and then streamed 
back to the host computer to be analyzed.

The IP core implementation is provided as the raw RTL files. The interface to the core is simple, and streams samples 
from the selected entropy source to whatever downstream logic is added by the user. This allows the user to have 
selectable entropy sources for the same application, such as backups if the health of an entropy source becomes 
degraded.

In addition, the TRNG Testbed IP core allows selection of desired clock frequency. Several TRNGs have a range of clocks
the source authors have identified as optimal, but the dynamic selection of clock frequency has the added benefit of
providing local, per-entropy-source optimization. 