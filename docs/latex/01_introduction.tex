\section{Introduction}

Secure true random number generation is an important problem for the security of communication, 
in order to ensure the fundamental requirements of confidentiality, integrity, availability, and
non-repudiation. Random numbers are used extensively in cryptography, such as during key generation in 
the recent post-quantum cryptographic public key encryption algorithm, ML-KEM. The widespread need for random
number generation has led to dozens of publications proposing implementations of entropy sources
based on metastability, jitter, thermal noise, and chaos, using a variety of entropy harvesting techniques. 
Meanwhile, many computing platforms on the market today are sold with built-in cryptoprocessors 
(and thus built-in TRNGs) in parallel with the performance platform, but these can often be proprietary 
with potential zero-day exploits hidden behind a black box for a determined and well-resourced attacker to find. 
Several recent strides have been made towards security through transparency in the hardware world; Raspberry Pi 
has recently conducted several security challenges inviting hardware security experts to attack the security of 
their new RP2350 microcontroller. In another example, OpenTitan is an impressive open-source root-of-trust (RoT)
project that is a model for the philosophy of security through transparency. Though bug bounties and ethical 
vulnerability disclosure is nothing new, transparency in chip security for commercial products improves consumer 
confidence in the chip they purchase. 

Furthermore, new true random number generators (TRNGs) and entropy sources based on metastability, 
jitter, chaotic boolean networks, and other sources are proposed, developed, and published at a 
rapid rate. Providing open access to a verified implementation of each new advancement allows 
security experts, consumers, regulators, and researchers to most quickly and easily verify and produce
new designs using these entropy sources. Existing work has been done already to make ring-oscillator (RO) based
TRNGs, emulation of them, and their analysis available to the community, but the work can be extended for 
integration into processor platforms, while also standardizing the interfaces to entropy sources to integrate 
into anywhere. For the benefit of the community, this work seeks to reduce the barrier to access by providing modeling,
hardware implementation, and analysis, as well as contribute to the open dialog of hardware security through transparency.

Foundational work by Petura et. al. produced an open-source catalog of six RO-based TRNGs, implemented such that 
they used the minimum possible components on high-end, low noise power supplies in order to ensure their measurements 
were as pure and free from deterministic noise as possible. From a design perspective, this is absolutely a reasonable
approach. However, each entropy source will eventually be integrated into a platform where other noise from nearby 
components is inescapable. The lifecycle of a given IP starts from an idea, is implemented into a proof of concept, and
then eventually implemented as a prototype before eventual initial release. This work and the proposed platform is 
intended to support the migration of proofs of concept into their eventual prototypes. Furthermore, to the benefit of 
the open-source community, a platform that targets common development boards like the Arty A7 or Numato Skoll allows a 
more widely accessible test device. All analysis and data collected with this platform of course comes with the caveat 
of elevated platform noise as compared to standalone proofs of concept.

\textit{Our contribution:}
\begin{enumerate}
    \item We provide a catalog of open-source entropy sources developed in recent literature.
    \item We explore, document, and demonstrate a variety of entropy generation techniques.
    \item We provide a playground of emulation models of each entropy generation technique.
    \item We package the catalog in a highly portable AXI-based interface for easy integration.
    %\item Build upon and amplify existing research to .
\end{enumerate}

This paper is organized into four sections: related research, evaluation, the catalog itself, and the analysis of random
data collected from each TRNG in the catalog. In the related reseach section, we discuss existing work done by Petura et.
al. and Pebay-Peroula et. al. providing the foundation for this paper. In the evaluation, we broadly discuss and cite our 
methods for modeling, implementing, and analyzing the collected catalog of TRNGs. Following evaluation, we discuss the 
specific details regarding our implementation of each entropy source in depth, from concept to integration. Finally, 
we analyze the TRNGs using the methods laid out in NIST SP800-22, -90b, and AIS 20/31.

% TODO: original implementations each get a tex file
% TODO: build overal demonstration platform (TrngTestbed?) that targets Arty A7 and Genesys 2
% TODO: tabulate implementation min entropy if listed, and provide captured min entropy values