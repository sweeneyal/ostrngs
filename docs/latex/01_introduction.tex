\section{Introduction}

Secure true random number generation is an important problem for the security of communication, 
in order to ensure the fundamental requirements of confidentiality, integrity, availability, and
non-repudiation. Random numbers are used extensively in cryptography, such as the recent
post-quantum cryptographic public key encryption algorithm, ML-KEM. The widespread need for random
number generation has led to dozens of publications proposing implementations of entropy sources
based on metastability, jitter, thermal noise, and chaos. This work seeks to catalog several of these 
entropy sources, discuss their implementations, and provide open access to all entropy sources in this 
catalog. In addition to the implementation source, a demonstration platform targeting common AMD FPGA 
platforms is provided in parallel for further evaluation of each entropy source's implementation. Finally, 
min-entropy estimates using NIST SP 800-90B on each entropy source are performed, provided, and discussed
at length.

Several recent strides have been made towards security through transparency in the hardware world; 
Raspberry Pi has recently concluded a first security challenge inviting hardware security experts
to attack the security of one-time-programmable (OTP) regions on their new RP2350 microcontroller.
They have since further announced a currently active security challege inviting more experts to 
perform side channel analysis (SCA) attacks on a hardened AES implementation. Though so-called bug 
bounties and inviting security experts probe new products is nothing new, transparency in chip 
security for commercial products improves consumer confidence in the chip they purchase. 

Furthermore, new rue random number generators (TRNGs) and entropy sources based on metastability, 
jitter, chaotic boolean networks, and other sources are proposed, developed, and published at a 
rapid rate. Providing open access to a verified implementation of each new advancement allows 
security experts, consumers, regulators, and researchers to most quickly and easily iterate and 
advance the capabilities of entropy sources to better meet their target requirements.

This paper seeks to provide firstly a catalog of open-source entropy sources developed in the 
literature, providing models and documentation for the implementation provided, and secondly 
a set of  automated, open-source, Python-based tools by which these entropy sources may be 
evaluated on a given commercially available AMD FPGA platform.

% TODO: original implementations each get a tex file
% TODO: build overal demonstration platform (TrngTestbed?) that targets Arty A7 and Genesys 2
% TODO: tabulate implementation min entropy if listed, and provide captured min entropy values