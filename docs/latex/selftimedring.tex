\subsection{Self-Timed Ring TRNG}

The self-timed ring (STR) TRNG proposed by Park et. al. is designed from the use of 
an oscillator that allows for simultaneous pulse propagation without encountering
collisions between pulses. The fundamental component at the heart of this TRNG is
the Muller C element [cite], designed such that it has the truth table listed 
in figure \ref{fig:mullerc_truth_table}.

\begin{figure}
    \label{fig:mullerc_truth_table}
\end{figure}

In comparison with the inverter ring oscillator-based TRNG, which has one "token"
that propagates through the entire inverter chain, the STR-TRNG allows several "tokens"
to propagate through the chain of Muller C elements. It is worth noting that the
implementation described in Park et. al. [cite] is designed as implemented in an ASIC. 

We reimplemented the work described in Park et. al. targeting an FPGA by implementing
the Muller-C element as an inferred latch. As the truth table in figure 
\ref{fig:mullerc_truth_table} shows, the Muller-C element has state memory, as well as signals
that cause it to change to different states. Therefore, the truth table can be implemented 
with a latch and basic logic elements.

This TRNG is designed such that the primary source of entropy is accumulated jitter of the
oscillator at the rising and falling edges. This jitter comes from thermal noise and thus 
follows a Gaussian distribution. Since every Muller C element's output is sampled and then
summed together to produce a single bit, the expectation is that several of these samples
will measure the oscillator's signal close enough to a transition point, where the jitter
will then ensure some randomness in these samples. This is an intential abuse of setup and 
hold timing constraints, instead fostering the often-undesirable non-determinism that timing
analysis seeks to avoid.

The implementation produced by the authors allows also an additional knob for configuring the
initial transient behavior of the ring; the Muller-C element has a precharge component, in this
implementation simply charging or discharging the inferred latch. The charge stored within the
Muller-C element changes the dynamics slightly, though eventually the self-timed oscillator is 
expected to approach its intrinsic waveform. 

Despite the entropy source being self-timed and thus autonomous, the boolean behavior is nonchaotic
and acts as a jittery oscillator rather than a high dimensional chaotic system.