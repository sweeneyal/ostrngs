\subsection{Mesh-Coupled XOR Autonomous Boolean Network TRNG}

The Mesh-Coupled XOR TRNG proposed by Lu et. al. is an autonomous boolean network designed 
based on first principles established by Gong et. al. that fundamentally uses several ring oscillators,
coupled XOR gates, and a mesh network topology to generate a chaotic-like entropy source. 
The autonomous boolean network mechanism, similar to work done by Rosin and Zhang and collaborators, 
allows for the use of nonlinear dynamics and chaos to expose the hardware fabric's manufacturing 
variance as an entropy source. 

The entropy source here is derived from the unpredictability of boolean chaos. As chaos is defined by
its sensitivity to initial conditions, the initial conditions are unknowable with infinite precision,
and as such, a security system with a chaotic entropy source can be assumed to be more secure. However,
proving chaotic behavior of a system is a hard problem, so no claims are made that chaos is guaranteed
present, though it is assumed due to its similarity to work by Rosin and Zhang.

The entropy source samples from several points within the boolean network, specifically at the left- and
right-hand sides of the coupled XOR units. Six sets of four measurement points are then defined, and each
of the four measurement points are XORed together to generate six bits of randomness.

\begin{figure}
\centering
\includegraphics[width=0.45\textwidth]{figures/drawio/meshcoupledxor_toplevel.png} % Adjust width as needed
\caption{Diagram of the Mesh Coupled XOR Boolean Network with the Coupled XOR (CX) Unit magnified.}
\label{fig:mesh_coupled_xor_diagram}
\end{figure}

As part of fully evaluating the entropy source, the model of the autonomous boolean network was recreated
both using the same modeling technique as Rosin et. al. and Gong et. al., and then with a new technique
that used slew rates as opposed to first order low-pass filtering. The recreation with the former technique
is provided in figure \ref{fig:mesh_coupled_xor_plde} and the latter technique in \ref{fig:mesh_coupled_xor_plsr}.

\begin{figure}
\centering
\includegraphics[width=0.45\textwidth]{figures/meshcoupledxor_plde.png} % Adjust width as needed
\caption{Recreated piecewise linear differential equation model of one node of Mesh Coupled XOR model.}
\label{fig:mesh_coupled_xor_plde}
\end{figure}

\begin{figure}
\centering
\includegraphics[width=0.45\textwidth]{figures/meshcoupledxor_slewrate.png} % Adjust width as needed
\caption{Piecewise linear slew rate model of one node of Mesh Coupled XOR model.}
\label{fig:mesh_coupled_xor_plsr}
\end{figure}