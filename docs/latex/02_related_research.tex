\section{Related Research}

In our research, we found that the terms "true random number generators (TRNGs)" and "entropy sources" were
often used interchangably. NIST's publication SP800-90a, -90b, and -90c discuss requirements and recommendations 
of sources of random bits, ranging from deterministic random bit generators (DRBGs), entropy sources, and random 
bit generators (RBGs). Meanwhile, AIS 20/31 uses the terminology deterministic random number generators (DRNGs), 
physical true random number generators (PTRNGs) and non-physical true random number generators (NPTRNGs). 
Additionally, DRBGs are often colloquially referred to pseudorandom number generators (PRNGs), and in the 
literature entropy sources are frequently colloquially referred to as true random number generators (TRNGs). 
As a result, we also refer to entropy sources as both entropy sources and TRNGs interchangably; however, in an 
analysis context especially with respect to NIST SP800-90b, we will use the terminology of "entropy source".

As part of the same philosophy and resulting movement promoting security through transparency, several publications 
have already made significant progress into catalogging and evaluating ring oscillator (RO) based TRNGs. Petura 
et. al. produced a catalog of six TRNGs, providing metrics for each for area, power consumption, entropy, efficiency, 
bit rate, and feasibility and repeatability. These metrics both indicate to users the potential of these entropy 
sources, while also indicating their pitfalls. Several of these entropy sources had relatively low bit rate 
($<$ 1 Mbits/s) which limits integration in high-end cryptographic platforms, while others consumed significant area, 
which limits integration in low-end microcontroller and tiny cryptographic platforms. Petura et. al. also provided a
reliability/feasibility score given to each TRNG, with a higher score indicating easily repeatable across devices, or
within different portions of a device, while lower scores indicated that increasing effort needed to be done to produce
repeatable results.

The catalog of TRNGs with their corresponding metrics provided a brochure-like approach of various proofs of concept,
informing the community of the pros and cons of the various TRNGs provided. The next natural step is to determine which 
TRNG suits a given application; in other words, which TRNG needs to evolve from a proof of concept to a prototype.
Pebay-Peroula et. al. took three of these TRNGs and built an evaluation platform and a set of Python analysis tools 
around them. The authors created a UART-based sampling structure that included components such as alarms and conditioning
units, creating a model implementation of a full TRNG prototype. 

The natural extension to the work of Petura et. al. and Pebay-Peroula et. al. is to provide more emulation, more 
hardware, and more analysis. The sample size of six predominantly ring-oscillator based TRNGs does not cover the 
full berth of existing research. Several other researched sources of entropy exist; metastable circuits are arguably 
one of the most widely experienced entropy sources, in that students learn early in their careers that setup and
hold violations and clock domain crossings that lead to metastability will turn their homework into random number 
generators.

In addition to metastability as an entropy source, chaos is often quoted as another source for entropy. Chaos has
made inroads into the hardware security space due to its intrinsic parallelism with cryptography and the promise of
physical uniqueness and entropy amplification, but many have created slow and insecure cryptographic primitives 
in a search for a crossing point, as Kocarev discusses in []. However, chaos has shown significant promise in the 
field of physically unclonable functions (PUFs) as well as in TRNGs. Work by Jiteurtragool et. al. created a simulated 
chaotic oscillator TRNG designed for an ASIC, where they leverage a unique transistor layout to create an inverse tent 
map, thereby stretching and compressing voltage activity to create a chaotic map []. Other papers have proposed unique 
topologies for coupling gates into autonomous boolean networks (ABNs) to produce high switching activity, such as 
Rosin, Lu et. al., and Addabbo et. al. 

Pebay-Peroula et. al. also contributed a significant step for the open-source community by both creating the foundational 
tools necessary to reduce barrier to entry, as well as philosophically establishing three core pillars to open source 
TRNG development. Emulation, hardware, and analysis are three key pillars that provide the community at large with the 
tools to understand, test, and evaluate TRNGs. Emulation informs understanding, hardware supports testing, and analysis 
inspires confidence; emulation and modeling of the physical source of entropy conveys the concept component demonstrated in
the hardware proof of concept while the analysis informs how well the proof of concept performs against theory. Providing
the implementation without the supporting model and analysis does not fully reduce the barrier to access. 