\subsection{Open-Loop Metastable TRNG}

The open-loop metastable TRNG is designed using two cascades of delay elements, one for the clock path, 
the other for the data path. The cascades are tuned such that they produce a single bit of randomness 
by forcing a flop at the end to be metastable. The contribution of Ben-Romdhane et. al. was
to implement a version of this concept targeting ASICs on a CMOS 65nm process node chip provided by 
STMicroelectronics \cite{ben-romdhane_design_2013}.

The primary mechanism for inducing metastability involves using two multiplexor chains
to calibrate the TRNG such that the delay induced through each multiplexor chain causes
several of the flops in the sampling flop chain to go metastable \cite{ben-romdhane_design_2013}. 
This is a similar concept to oscillator-based TRNGs, as they ideally capture transition points within the
oscillator signal, except that the premise here is less about the jitter of the signal,
but rather the flop's internal sampling structure settling finally on '0' or '1'. 

The final implementation given by the authors did not pass all the given tests in the 
AIS-31 standard statistical battery, and only one variant passed eight of the nine 
tests, failing the entropy estimation test \cite{ben-romdhane_design_2013}.

For the implementation given here, some liberties were taken in the implementation; 
namely, the control of the coarse chains was automated by an internal set of counters
that identified and scored the most metastable combination of the coarse chains, thereby
identifying the most number of unstable flops. Each flop in the fine stage is provided an 8-bit 
counter that increments whenever the flop measures a '1', over 255 consecutive samples. When 
255 consecutive samples have been collected, the counters are then evaluated such that if the 
attached flop counter measures close to 127 (because $127/255\approx0.5$) the flop is evenly 
metastable. If the flop measures too high or too low, the flop is unevenly metastable, or possibly
even stable if the count measures 0 or 255 exactly. The architecture diagram is provided in 
figure \ref{fig:openloopmeta_health}.

\begin{figure}
\centering
\includegraphics[width=0.45\textwidth]{figures/drawio/openloopmeta_health.png} % Adjust width as needed
\caption{Architecture diagram of integrated health sensor for open-loop metastable entropy source.}
\label{fig:openloopmeta_health}
\end{figure}

By the end of an initial measurement period, the best occurring configuration is accepted and used for
the entropy source. Ideally, the configuration with the most metastable flops used in the final 
XOR summation at the entropy source's output theoretically allows for better min-entropy. 
Additionally, the MUX21 stages implemented in the ASIC are in this case retargeted to 
AMD-Xilinx LUTs, and the buffer elements between the measuring flops in the fine measurment 
chain are similarly retargeted to LUTs.