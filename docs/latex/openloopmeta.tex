\subsection{Open-Loop Metastable TRNG}

The open-loop metastable TRNG is designed using two cascades of delay elements, one for the clock path, 
the other for the data path. The cascades are tuned such that they produce a single bit of randomness 
by forcing a flop at the end to be metastable. Several implementations of this concept have been 
implemented within the literature at [add sources here], but the contribution of Ben-Romdhane et. al. was
to implement a version of this concept targeting ASICs on a CMOS 65nm process node chip provided by 
STMicroelectronics.

The primary mechanism for inducing metastability involves using two multiplexor chains
to calibrate the TRNG such that the delay induced through each multiplexor chain causes
several of the flops in the sampling flop chain to go metastable. This is a similar 
concept to oscillator-based TRNGs, as they ideally capture transition points within the
oscillator signal, except that the premise here is less about the jitter of the signal,
but rather the flop's internal sampling structure settling finally on '0' or '1'. 

The final implementation given by the authors did not pass all the given tests in the 
AIS-31 standard statistical battery, and only one variant passed eight of the nine 
tests, failing the entropy estimation test.

For the implementation given here, some liberties were taken in the implementation; 
namely, the control of the coarse chains was automated by an internal set of counters
that identified and scored the most metastable combination of the coarse chains, thereby
identifying the most number of unstable flops. Having several metastable flops incorporated
into the final XOR summation theoretically allows for better min-entropy. Additionally, the
MUX21 stages implemented in the ASIC are in this case retargeted to AMD-Xilinx LUTs, and the
buffer elements between the measuring flops in the fine measurment chain are similarly 
retargeted to LUTs.