\section{Evaluation of Entropy Sources}

In keeping with the precedent set by Pebay-Peroula et. al., the entropy sources collected in this catalog will
be provided with emulation, hardware, and analysis. Additionally, due to AMD-Xilinx's significant dominance over the 
FPGA market, several TRNGs using generic concepts have been developed using primitives specific to AMD-Xilinx FPGAs. 
Often specific implementations require a specific primitive unavailable in other platforms, or have been intentionally 
optimized to reduce utilization or match the paper's specification.

\subsection{Emulation}

Each entropy source should have some form of emulator or model with which to analyze its behavior and to compare
results against. For many such entropy sources, the natural default modeling technique is SPICE modeling. 
SPICE modeling is frequently used for simple TRNG circuit simulation, with parameters modified to see what effect the 
unique hardware characteristics would produce, and thereby demonstrating the circuit as an entropy source. However,
this modeling technique is not only complicated but slow, and so often other emulation techniques are used as appropriate
to the entropy source.
% Need to include reasons why and why not to go with SPICE modeling, e.g. for large networks or for certain entropy
% sources.

Pebay-Peroula et. al. provided several emulation techniques for the three supported TRNGs with emulated thermal and 
flicker noise. % Discussion of their emulation techniques here.

For chaotic TRNGs, several techniques exist in the literature. For example, Ghil and Mulhaupt's boolean delay equations
are used to model longer-time-scale, discrete interactions between gates. In another example, Lu et. al. successfully used
the techniques proposed by Gong et. al. to model their autonomous boolean network at a smaller time resolution and 
mixed-signal/analog perspective. The modeling technique used within our paper for chaotic oscillators was proposed by 
Sweeney et. al. for more accurate analog simulation of autonomous boolean networks.
% Staple Sweeney et. al. here, if we want to publish it as part of this paper.
% Otherwise we need to guide readers to refer to that paper for discussion of the pros and cons of that model.

\subsection{Hardware Implementation and IP Core}

Pebay-Peroula et. al. proposed a TRNG evaluation platform based on UART, an internal command processor, and a FIFO.
Their hardware implementation provides a clear foundation on which to explore TRNGs. However, two shortcomings remain that
reduce its ease of integration; firstly, their PTRNG IP has unique self-defined interface signals and secondly, the UART 
interface is fundamentally limited by the bit rate. 

Regarding the first point, having a unique self-defined interface creates an unintentional learning curve that increases
barrier to entry. If one were to use their PTRNG IP as is, this is not an issue, but if one wanted to integrate their IP
as part of another, e.g. as a cryptographic hardware accelerator, then the learning curve rears its ugly head. In fact, 
this is fundamentally why standard buses like AXI or Wishbone exist; standardized interfaces allow easy integration with
other IP.

The foundational work done by Petura et. al. provides a huge step forward on behalf of the open source and hardware security
communities; however, the same self-defined interface issue applies here. Their techniques for measurement and implementation
also create a barrier to entry, as they do not provide a standard interface to their entropy sources for integration into other
platforms. For example, Pebay-Peroula et. al. needed to reimplement three of their TRNGs to integrate them into their work;
this duplication of effort partly defeats the purpose of open-source. At a fundamental level, every entropy source needs to be 
verifiable as a standalone implementation as well as be integrable into a given platform. Thus, all of our entropy sources have
the same set of interfaces tabulated in Table \ref{tab:interface_definitions}.

\begin{table}
    \centering
    \caption{Fundamental TRNG Interface Definitions}
    \begin{tabular}{|l|l|l|}
        \hline
        Signal    & Type               & Description \\
        \hline
        i\_clk    & std\_logic         & Sampling clock \\
        i\_resetn & std\_logic         & Reset signal \\
        o\_rng    & std\_logic\_vector & Bus of source-dependent width \\
        o\_valid  & std\_logic         & Indicator of bus valid   \\
        \hline
    \end{tabular}
    \label{tab:interface_definitions}
\end{table}

Regarding the second point, in their provided source code, they have the UART baud rate set to 115200 baud. Of course, 
one can increase this parameter, but common FPGA platforms such as the Arty A7 have low data output rate compared to 
modern TRNGs, even if the onboard FTDI chip is reprogrammed to support its maximum baud rate of 12 Mbaud. At 12 Mbaud, 
the effective data rate is 1.2 Mbytes/s, while some TRNGs are capable of up to 400 Mbytes/s. Therefore, the fundamental
problem becomes one of throughput; for some types of analysis, up to a million samples need to be collected, and many 
FPGAs do not have enough in-chip memory to store the amount of samples necessary. Therefore, the FIFO needs to be 
directed not towards the UART interface but rather towards another downstream memory component, often a RAM interface.

Our platform, named TRNG Testbed, thus aims to do the following:
\begin{enumerate}
    \item Provide a standard interface to all entropy sources, so that testing and integration is seamless both with 
        this platform and any future platform.
    \item Provide a parameterized IP core to allow guided instantiation of supported entropy sources.
    \item Leverage industry-standard interfaces to make IP core compatible with modern platforms.
    \item Provide an example platform roughly equivalent to existing test platforms with extensions to support further testing and analysis.
\end{enumerate} 

The TRNG Testbed IP core provides several configuration options at synthesis-time and at run-time. Firstly, a 
configurable number of different types of supported entropy sources can be instantiated through the top level generics.
By making the entropy source lineup configurable, users can instantiate several different types of entropy sources,
or several copies of the same entropy source. This design choice gives users the freedom to perform side-by-side comparison 
between two different entropy sources in a given build, or identifying the upper and lower bound of performance for a 
sample set  of the entropy source. In addition, the IP core allows selection of desired clock frequency. Several 
TRNGs have a range of clock frequencies the source authors have identified as optimal, thus the dynamic selection of clock 
frequency has the added benefit of providing local, per-entropy-source optimization, as well as the ability to characterize 
the entropy source over a range of frequencies.

Further, the IP core is designed for integration into a processor-based system on FPGA. The vast majority of TRNG 
applications are within processor-based designs, usually as a part of an algorithm such as key generation within 
cryptography like ML-KEM, simulations of random phenomena, etc. As a result, the IP core TRNG Testbed is designed with 
an AXI4 Lite interface, maintaining compatibility with AMD Vivado block design workflow. Additionally, because the 
TRNG Testbed is designed for entropy source instrumentation, an additional master AXI interface was added in order to 
provide a direct memory access (DMA) feature, allowing the TRNG Testbed to write a configurable number of samples into a
downstream memory region for readout and use by a processor.

\subsection{Analysis}

Many analysis techniques exist and are commonly used throughout the literature to evaluate TRNGs. Often, publications 
proposing new TRNG architectures often present their results in reference to a suite of different mathematical 
evaluation tools, such as DIEHARD and DIEHARDER, NIST SP800-22 and -90b, and AIS-20/31 to name a few. Petura et. al. 
present a collection of TRNGs that meet AIS-20/31 specification, while Pebay-Peroula et. al. implement their own analysis
techniques in Python. Meanwhile several new TRNG architecture proposals use the NIST SP800-90b tests.

% Discuss what all entropy assessment tools are used in this paper. We should integrate the NIST tests into a pybind11 
% implementation, thereby allowing us to use their source code directly. We should also integrate Pebay-Peroula et. al.'s 
% analysis for use with our platform.


\section{NIST SP 800-90B}

The NIST Special Publication 800-90B is the second in a series of publications regarding the design and implementation
of random bit generators (RBGs). Whereas SPs 800-90A and 800-90C describe the design and validation of deterministic 
random bit generators and their integration with an entropy source, SP 800-90B provides specification for the design
and validation of entropy sources. 

Central to the recommendation set forth by NIST SP 800-90B is the measurement of entropy; specifically the publication
focuses on min-entropy. 

Furthermore, the publication sets out a block diagram model of an expected design of an entropy source, which includes
the following components:

\begin{enumerate}
    \item An analog noise source.
    \item A digitization mechanism to convert the noise source's output into bits.
    \item Built-in health tests with error reporting.
    \item An optional conditioning mechanism.
\end{enumerate}

The publication additionally describes a high-level functional interface that specify several example interactions that
the system can perform. These are not given as requirements, since entropy sources may vary in implementation.

\begin{enumerate}
    \item \textbf{GetEntropy} - a mechanism to retrieve an amount of entropy.
    \item \textbf{GetNoise} - a mechanism to retrieve raw, post-digitization samples of the noise source for validation 
        or health testing.
    \item \textbf{HealthTest} - a mechanism to request that the entropy source perform health tests.
\end{enumerate}

\subsection{Data Collection and Validation}

The publication specifies three requirements regarding data collection, first two are mutually exclusive depending on 
the inclusion of a conditioning component, but essentially require the collection of 1 000 000 samples for validation.
The third requirement specifies that for the restart tests, 1000 consecutive samples must be collected, 1000 times. 
This allows for testing for undesired correlation between ideally unique rows and columns.

% Note: revisit independent and identically distributed (IID) vs non-IID, not sure I understand this right this second.
% Does IID essentially mean independent and uniformly distributed?
The specification further breaks down the validation testing into the \textit{independent and identically distributed} 
(IID) path and the non-IID path. For an entropy source to be independent and identically distributed, the symbols it
generates need to be distributed according to a uniform distribution, that is, no symbol can be more commonly occurring
than any other, and each symbol must be independent of any other symbol.

\subsection{Determining Validation Pathway}

In order to determine the correct path to take for validation, the noise source must be tested to see whether the IID
assumption holds. Several tests are provided and are categorized into two categories, namely permutation testing and 
additional chi-square statistical tests. Permutation testing is a means to test a statistical hypothesis by comparing
the actual value of a test statistic against inferred reference distributions generated by the input data. 
Additionally, the chi-square tests are provided in order to discover dependencies in the probabilities between
successive samples or discover discrepancies between the distributions of different subsets of the dataset and the
dataset as a whole.

The permutation tests are listed as follows:

\begin{enumerate}
    \item Excursion Test Statistic
    \item Number of Directional Runs
    \item Length of Directional Runs
    \item Number of Increases and Decreases
    \item Number of Runs based on the Median
    \item Length of Runs based on the Median
    \item Average Collision Test Statistic
    \item Periodicity Test Statistic
    \item Covariance Test Statistic
    \item Compression Test Statistic
\end{enumerate}

% Add notes on each test here as they are implemented

The independence and goodness-of-fit tests depend on the type of input data, as the mechanisms change based on whether
the range of outputs from a given noise source is binary or non-binary.

\subsection{Validation Tests}

The primary metric the NIST SP 800-90B tests estimate is the min-entropy metric, which fundamentally describes how 
difficult it is for an adversary to predict the output of a given entropy source. As a result, several tests are 
provided depending on whether the IID assumption has been shown to be valid or not. If the samples a noise source
produces is shown to be independent and identically distributed, then the number of occurrences of each possible 
symbol should be roughly equivalent in a given dataset. However, if the noise source is shown to not be IID, then some
types of validation tests will give overconfident estimates of min-entropy. 

The min-entropy estimators provided are listed as follows:

\begin{enumerate}
    \item Most Common Value Estimate
    \item Collision Estimate (binary inputs only)
    \item Markov Estimate (binary inputs only)
    \item Compression Estimate (binary inputs only)
    \item \textit{t}-Tuple Estimate
    \item Longest Repeated Substring (LRS) Estimate
    \item Multi Most Common in Window Prediction Estimate
    \item Lag Prediction Estimate
    \item MultiMMC Prediction Estimate
    \item LZ78Y Prediction Estimate
\end{enumerate}

\subsection{Health Tests}

% Note: revisit health tests here